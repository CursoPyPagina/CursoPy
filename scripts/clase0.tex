\documentclass[12pt]{article}
\usepackage[utf8]{inputenc}

\title{Prueba}
\author{luisapaez31416 }
\date{July 2022}


\setlength{\parindent}{0pt}
\begin{document}

\maketitle
% Esto es un comentario. Solo se ve en el archivo .tex pero no se 
% procesa en el documento PDF, es decir, en el PDF no se vera esto
% que hemos escrito. Suelen ser utiles para descibir lo que estamos
% escribiendo y haciendo dentro de nuestro archivo .tex

% Esto es una seccion, le hemos cambiado el nombre
\section{Poemas de Lope de Vega}

% Poema. El titulo del poema ira en negritas
% Para escribir un texto en negritas utilizamos el comando \textbf{}
\textbf{Ir y quedarse, y con quedar partirse}
 
Ir y quedarse, y con quedar partirse,
partir sin alma y ir con alma ajena,
oír la dulce voz de una sirena
y no poder del árbol desasirse;
% al dejar un espacio en blanco indicamos que se debe escribir el siguiente
% texto en la siguiente línea

arder como la vela y consumirse
haciendo torres sobre tierna arena;
caer de un cielo, y ser demonio en pena,
y de serlo jamás arrepentirse;

% podemos centrar el texto utilizando un entorno de centrado 
% los entornos siempre empiezan con begin y terminan con end
\begin{center}
hablar entre las mudas soledades,
pedir pues resta sobre fe paciencia,
y lo que es temporal llamar eterno;
\end{center}

% texto en cursiva \textit{}
creer sospechas y negar \textit{verdades,
es lo que llaman en el mundo ausencia},
fuego en el alma, y en la vida infierno.


% pagina nueva
\newpage

% Agregamos otra seccion
\section{Expresiones Matemáticas}

% los exponentes los escribimos como ^{} y los subindices como _{}

Mi primer ecuación: $f(x) = x^{2}_{1} + 2x_{2}+5$

% ecuacion de Euler
% el comando \pi arroja
% el simbolo matematico de pi
% podemos maner el espacio vertical entre lineas
\vspace{0.5cm}

Ecuación de \textit{Euler}:
$e^{i \pi} + 1 = 0$

Hablamos de la ecuación más bella de las Matemáticas, la cual está dada por $e^{i \pi} + 1 = 0$
\vspace{0.5cm}

Fórmula de la chicharronera:
$$
x = \frac{-b\pm \sqrt{b^{2}-4ac}}{2a}
$$

Otra forma de escribir expresiones matemáticas

% \left(  hace que el tamaño del parentesis ( se adapte a lo que estamos escribiendo
% \right(  hace que el tamaño del parentesis ) se adapte a lo que estamos escribiendo
% el entorno array nos permite escribir matrices, donde cc nos indica que la matriz
% tendra dos columnas, y el numero de filas que deseemos. Si ponemos ccc la matriz 
% tendra 3 columnas
% & separa lo que ira en una entrada con lo que ira en la otra entrada de la fila
% \\ separa una fila de otra

\begin{equation}
    M= \left(\begin{array}{cc}
        1 & 0 \\
        0 & 1 
    \end{array}\right)
\end{equation}

\newpage
\section{Listas}

% subseccion
\subsection{Listas básicas}

Crearemos nuestra primer lista, para ello

% dentro del entorno itemize, cada item es un punto de la lista:
\begin{itemize}
    \item Este es un punto de la lista
    \item Este es otro punto de la lista
\end{itemize}

De forma análoga tendremos lista numeradas, para ello utilizaremos el entorno
\textit{enumerate}:
\begin{enumerate}
    \item Hola
    \item Mundo
\end{enumerate}
\end{document}
