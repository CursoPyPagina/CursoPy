\documentclass[12pt]{article}

% Paquetes
\usepackage{mathptmx}
\usepackage{anysize}
\usepackage[utf8]{inputenc}
\usepackage[T1]{fontenc}
\usepackage[all]{xy}
\usepackage[spanish, mexico]{babel}
\usepackage{tikz}
\usepackage{graphicx}
\usepackage{subcaption}
\usepackage{amsmath}
\usepackage{amssymb}
\usepackage{setspace}
\usepackage{cancel}
\usepackage{enumerate}
\usepackage{amsthm}
\usepackage{bookmark}
\usepackage{amsfonts}
\usepackage{multirow}
\usepackage{multicol}
\usepackage{mathrsfs}
\usepackage{vmargin}
\usepackage{euscript}
\usepackage{xcolor}
\usepackage{fancyhdr}
\usepackage{subfiles}
\usepackage{stmaryrd}
\usepackage{listings}

% Portada
\title{Ejercicio para prácticar}
\author{Luis Fernando Apáez Álvarez}
\date{05 de Agosto del 2022}

% Margen
\setmargins{2cm}% margen izquierdo
{1.5cm}% margen superior
{16.5cm}% anchura del texto
{23.42cm}% altura del texto
{10pt}% altura de los encabezados
{1cm}% espacio entre el texto y los encabezados
{0pt}% altura del pie de página
{2cm}% espacio entre el texto y el pie de página

% sangrado
\setlength{\parindent}{0pt}
\setlength{\headheight}{17pt}


\begin{document}
\maketitle
\textbf{\large{Solución completa:}} 
\vspace{1cm}

\textbf{\large{Ejercicios}}

%%%%%%%%%%%%%%%%%%%%%%%%%%%%%%%%%%%%%%%%%%%%%%%%%%%%%%%%%%%%%%%%%%%%%%%%%%%%%%%%%%%%%%%%%%%%%%%%%%%%%%%%%%%%%%%%%%
%%%%%%%%%%%%%%%%%%%%%%%%%%%%%%%%%%%%%%%%%%%%%%%%%%%%%%%%%%%%%%%%%%%%%%%%%%%%%%%%%%%%%%%%%%%%%%%%%%%%%%%%%%%%%%%%%%
%%%%%%%%%%%%%%%%%%%%%%%%%%%%%%%%%%%%%%%%%%%%%%%%%%%%%%%%%%%%%%%%%%%%%%%%%%%%%%%%%%%%%%%%%%%%%%%%%%%%%%%%%%%%%%%%%%
%%%%%%%%%%%%%%%%%%%%%%%%%%%%%%%%%%%%%%%%%%%%%% Codigo solucion %%%%%%%%%%%%%%%%%%%%%%%%%%%%%%%%%%%%%%%%%%%%%%%%%%% 

\begin{enumerate}
    \item[] Texto 1: El matemático y físico suizo \textit{Leonhard Euler} nació un 15 de abril, concretamente de 1707, y es considerado el principal matemático del siglo \textbf{XVIII} y uno de los más grandes y prolíficos de todos los tiempos. A él le debemos la ecuación, considerada como ``la más bella de las Matemáticas'', dada por
    \begin{equation}
        e^{i\pi}+1=0
    \end{equation}
    donde $i=\sqrt{-1}$ es la unidad imaginaria, $\pi\approx3.14159265356$ la famosa constante de \textit{\textbf{Arquímedes}} y $e\approx2.7182818$ la constante de \textbf{\textit{Euler}}.
    \item[] Texto 2:\\ \textbf{Ejercicio de Clase:} Resolver el siguiente sistema de ecuaciones:
    \begin{align*}
        &(1)\ \  x + y = 7\\
        &(2)\ \  2x - y= -1
    \end{align*}
    \textbf{Solución:} De la ecuación $(1)$ despejamos $x$:
    $$
    x+y=7 \ \ \Rightarrow \ \ x = 7-y
    $$
    Luego, el valor obtenido anteriormente lo sustituimos en la ecuación $(2)$:
    $$
    2x-y=-1 \ \ \Rightarrow \ \ 2(7-y)-y=-1
    $$
    de donde
    $$
    -1=2(7-y)-y=14-2y-y=14-3y \ \ \Rightarrow \ \ -15=-3y \ \ \Rightarrow y=5
    $$
    De lo anterior concluimos que
    $$
    x = 7-y=7-5=2
    $$
    entonces la solución es $x=2$ y $y=5$.
    \item[] Texto 3:\\
    Pasos para resolver una ecuación de grado $\mathbf{2}$. Dada una ecuación $ax^{2}+bx+c=0$, debemos:
    \begin{enumerate}
        \item[1.] Ver el comportamiento del determinante $b^{2}-4ac$ de la ecuación. Donde
        \begin{itemize}
            \item Si el determinante es menor a cero, las soluciones de la ecuación serán complejas (es decir, las soluciones estarán en $\mathbb{C}$).
            \item Si el determinante es cero, entonces la ecuación tiene sólo una solución y es real.
            \item Si el determinante es mayor a cero, entonces la ecuación tiene dos soluciones reales (es decir, las soluciones estarán en $\mathbb{R}$).
        \end{itemize}
    \item[2.] Una vez determinando el \textit{comportamiento del determinante}, procedemos a resolver la ecuación por cualquiera de los siguientes \textbf{métodos:}
        \begin{itemize}
            \item Factorización.
            \item Fórmula general.
            \item Completando el \textbf{TCP}
        \end{itemize}
    \end{enumerate}

    \item[] Texto 4:\\
    \textbf{Évariste Galois} fue un matemático francés del siglo \textbf{XIX}, el cual sentó las bases para el desarrollo de una nueva teoría matemática, la cual lleva su nombre (\textit{Teoría de Galois}). Siendo aún muy joven, resolvió un problema matemático abierto, dando la condición necesaria y suficiente para que una ecuación algebraica sea resuelta por radicales.
    
    La idea básica de su hallazgo es, si consideramos que toda ecuación de grado dos puede resolverse mediante una fórmula general, ¿qué podemos decir respecto a las ecuaciónes de \textbf{grado tres} ($ax^{3}+bx^{2}+cx+d=0$)? 
    
    En realidad también hay una versión de la \textit{chicharronera} para resolver ecuaciones de grado tres. Más aún, para las ecuaciones $ax^{4}+bx^{3}+cx^{2}+dx+e=0$ de grado 4, también hay una versión de la \textit{chicharronera}. La pregunta natural es ¿hay una especie de \textit{chicharronera} para ecuaciones de grado 5? o de manera más general, ¿hay una especie de \textit{chicharronera} para ecuaciones de grado mayor o igual a 5?
    
    Lo que probo el joven Galois fue que, en realidad, no puede existir una fórmula general, o una especie de \textit{chicharronera}, para ecuaciones de grado mayor o igual a 5.
    
    \item[] Texto 5:\\
    Sabemos que la matriz identidad de $2\times 2$ está dada por
    \begin{equation}
        I_{2\times2}=\left(\begin{array}{cc}
             1& 0 \\
             0& 1
        \end{array}\right)
    \end{equation}
    Ahora, dada la matriz $M=\left(\begin{array}{cc}
             2& 3 \\
             1& 1
        \end{array}\right)$
    podemos efectuar la multiplicación $MI_{2\times2}$ e $I_{2\times2}M$ para notar que
    \begin{align*}
        MI_{2\times2}=\left(\begin{array}{cc}
             2& 3 \\
             1& 1
        \end{array}\right)\left(\begin{array}{cc}
             1& 0 \\
             0& 1
        \end{array}\right)=\left(\begin{array}{cc}
             2\cdot 1+3\cdot 0&2\cdot 0+3\cdot 1   \\
             1\cdot 1+1\cdot 0& 1\cdot 0+1\cdot 1
        \end{array}\right)=\left(\begin{array}{cc}
             2& 3 \\
             1& 1
        \end{array}\right)
    \end{align*}
    con lo anterior podemos ver el porqué se le denomina \textbf{matriz identidad.} 
\end{enumerate}

\end{document}